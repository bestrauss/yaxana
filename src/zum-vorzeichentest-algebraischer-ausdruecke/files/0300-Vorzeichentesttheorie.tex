%
% LaTeX
%
%
\pagebreak
\section{Eine Vorzeichentesttheorie}


\begin{remark}[Aufgabenstellung]\label{re:Aufgabenstellung}
Sei der reelle Wert $\xi_E$ eines Ausdrucks $E$ ganzalgebraisch
(der allgemeine Fall läßt sich darauf zurückführen, \cite{BFMS, BFMSS}).
Seien $z_k$ die $N\in\N$ Nullstellen
des Strukturpolynoms $p_E(z)$ von $E$,
die entweder reell sind 
oder in konjugiert komplexen Paaren auftreten,
dann gilt für das Produkt der Nullstellen
\begin{equation*}
  \prod_{k=1}^N z_k = c_0 \in \Z
\end{equation*}
und somit insbesondere auch
\begin{equation*}
  \xi_E\ne0
  \quad\Rightarrow\quad
  \prod_{k=1}^N |z_k| \ge 1.
\end{equation*}
Nun ist der Wert $\xi_E$ des Ausdrucks $E$
eine der Nullstellen $z_k$.
Sei o.B.d.A. $\xi_E = z_1$.
Dann ergibt sich
\begin{equation}\label{eq:Nulltrennungschranke}
  \xi_E\ne0
  \quad\Rightarrow\quad
  |\xi_E|
  \ge
  \frac{1}{\prod_{k=2}^N |z_k|}
  \ge
  \frac{1}{(\max_k|z_k|)^{N-1}}.
\end{equation}
Der Term $(\max_k|z_k|)^{-(N-1)}$ wird als Nulltrennungsschranke 
für ganzalgebraische Werte verwendet \cite{BFMS}.

Um das Vorzeichen eines gegebenen, 
bekanntermaßen nicht verschwindenden Wertes 
eines Ausdrucks $E$ zu ermitteln,
kann man Näherungen $\hat{\xi}_E$ mit schrittweise wachsender 
absoluter Genauigkeit berechnen,
bis man eine nicht verschwindende Näherung hat, 
an der sich das Vorzeichen dann ablesen läßt.
Verschwände der Wert aber, dann berechnete der Algorithmus
immer mehr Nullen hinter dem Komma, ohne zu terminieren.
Die Nulltrennungsschranke erlaubt es nun, 
die Rechnung bei einer entsprechenden Genauigkeit abzubrechen,
wonach dann $\sgn\xi_E=\sgn\hat{\xi}_E$ gilt.

Es handelt sich bei Un-\refequation{eq:Nulltrennungschranke}
in der typischen Praxis um eine unglaublich miserable Abschätzung,
aber im schlimmsten Fall könnten ja tatsächlich alle Nullstellen
mehr oder weniger genau auf einem riesigen Kreis 
um den Ursprung der komplexen $z$-Ebene
mit dem Radius $\max_k|z_k|$ liegen,
während nur eine einzige Nullstelle 
in der Gegend von $|c_0|(\max_k|z_k|)^{-(N-1)}$
all diese Nullstellen kompensiert.

Was man viel lieber hätte,
wären hingegen Nullstellenpaare mit $|z_{k_0}\cdot z_{k_1}|=1$,
 so wie das bei gewissen zeitdiskreten Systemen
 in der Nachrichtentechnik vorkommt;
 man denke an die Übertragungsfunktionen 
 der Analyse-Resynthese-Kanäle
 in perfekt rekonstruierenden Polyphasen-
 oder Quadraturspiegelfilterbänken;
 ähnlich auch den Polstellen-Nullstellen-Paaren von Allpaßfiltern.
Unter solch himmlisch geordneten Umständen könnte man die Abschätzung
\begin{equation}\label{eq:Nulltrennungsschranke}
  \xi_E\ne0
  \quad\Rightarrow\quad
  |\xi_E|
  \ge
  \frac{1}{\max_k|z_k|}
\end{equation}
verwenden, in der der monströse Exponent $N-1$ fehlt.
Wie aus den Bildungsregeln des Strukturpolynoms hevorgeht,
ist $N$ das Produkt der Wurzelexponenten 
sämtlicher Wurzeloperationen in $E$,
und daher der Grund,
weshalb sich bisher die Behandlung von Ausdrücken 
mit bereits moderat vielen Wurzeloperationen 
in der Praxis verboten hat.
Weiterhin wäre zusätzlich auch fest $c_0=1$,
so daß nicht ein womöglich viel größerer Koeffizient
mit bloß $1$ nach unten abgeschätzt würde.
\end{remark}


\begin{definition}[Wohlgeratener Ausdruck]
Wir nennen einen Ausdruck $E$ 
mit dem nicht-verschwindenden Wert $\xi_E$ {\em wohlgeraten},
wenn in der Menge der Nullstellen
seines Strukturpolynoms $p_E(z)$
in der komplexen $z$-Ebene 
neben Nullstellenpaaren, 
deren Produkte jeweils den Betrag $1$ haben,
nur solche nicht-paarigen Nullstellen vorkommen,
deren Wert selbst den Betrag $1$ hat.
\end{definition}


\begin{theorem}[Hauptsatz]\label{th:Hauptsatz}
Seien $E_0$ und $E_1$ zwei 
(gemäß \refdefinition{de:Konjugierter Ausdruck})
{\em konjugierte} Ausdrücke,
deren Werte nicht verschwinden,
dann ist der Ausdruck
$E_0\div E_1$ (lies: $E_0\times(1 \div E_1)$) wohlgeraten.
\begin{proof}
Das Strukturpolynom $p_{E_0\div E_1}(z)$
des Ausdrucks $E_0\div E_1$ wird 
(gemäß \refdefinition{de:Strukturpolynom}) gebildet,
indem [da die Ausdrücke konjugiert sind 
und daher $p_{E_0}(z)=p_{E_1}(z)$ gilt]
jede Nullstelle von $p_{E_0}(z)$
durch sich selbst und durch 
jede andere Nullstelle von $p_{E_0}(z)$
geteilt wird.
Man erhält einerseits Nullstellen $\frac{z_{k}}{z_{k}}=1$
und andererseits Nullstellenpaare
$\frac{z_{k}}{z_{m}}$ und $\frac{z_{m}}{z_{k}}$
mit $\frac{z_{k}}{z_{m}}\frac{z_{m}}{z_{k}}=1$.
\end{proof}
\end{theorem}


\begin{definition}[Testausdruck]\label{de:Testausdruck}
Sei $\xi_E$ der Wert eines Ausdrucks $E$.
Wir bezeichnen den Ausdruck $E'$ mit dem Wert $\xi_{E'}$
als einen {\em Testausdruck für $E$},
wenn $\sgn\xi_E = \sgn\xi_{E'}$.
\end{definition}


\begin{definition}[Wohlgeratener Testausdruck]
Sei $E'$ ein Testausdruck für $E$
und sei $E'$ für den Fall $\xi_E\ne 0$ wohlgeraten,
dann nennen wir $E'$ einen wohlgeratenen Testausdruck für $E$.
\end{definition}


\begin{theorem}[Ein wohlgeratener Testausdruck]\label{th:Ein wohlgeratener Testausdruck}
Seien Ausdrücke $E_0$, $E_1$ und $E=E_0-E_1$
und gelte für die Werte $\xi_{E_0}\ne 0$ und $\xi_{E_1}\ne 0$,
dann ist 
\begin{equation*}
  E' = \frac{E_0-E_1}{|E_0|+|E_1|}.
\end{equation*}
ein wohlgeratener Testausdruck für $E$.
\begin{proof}
Da der Wert des Nenners
aufgrund der Voraussetzungen inklusive der Absolutwertbildungen
positiv ist, gilt $\sgn E'=\sgn E$ 
weshalb $E'$ nach \refdefinition{de:Testausdruck}
ein Testausdruck für $E$ ist.
Weiterhin sind im Fall $\xi_{E_0}\ne\xi_{E_1}$
bzw. $\xi_E\ne 0$, Zähler und Nenner 
konjugierte Ausdrücke gemäß \refdefinition{de:Konjugierter Ausdruck},
d.h. Zähler und Nenner haben dasselbe Strukturpolynom,
denn laut \refdefinition{de:Strukturpolynom}
gilt bei der Bildung des Strukturpolynoms 
\begin{equation*}
p_{-E}(z) = p_{|E|}(z) = p_E(z),
\end{equation*}
weshalb der Bildungsvorgang beider Strukturpolynom 
in jedem Zwischenschritt für Zähler und Nenner 
dasselbe Ergebnis liefert.
Laut Haupt-\reftheorem{th:Hauptsatz}
ist damit der Ausdruck $E'$ wohlgeraten.
\end{proof}
\end{theorem}

\pagebreak
\begin{algorithm}[ZVAA Vorzeichentest]\label{al:Vorzeichentest}
Bestimmung des Vorzeichens des Wertes $\xi_E$ 
eines Ausdrucks $E=E_0-E_1$.
\begin{enumerate}
\item Bilde den wohlgeratenen Testausdruck $E'$
      nach \reftheorem{th:Ein wohlgeratener Testausdruck}.
\item Berechne $\sgn\xi_{E'}$
      per {\tt BFMSS[2]} wie in \cite{BFMSS, PIYAP} beschrieben,
	  wobei allerdings in dem Term für die Nulltrennungsschranke
	  der Exponent $N_{E'}-1$ (der dort $D(E')-1$ genannt wird)
	  durch $1$ ersetzt wird.
\item {\tt return $\sgn\xi_{E'}$}.
\end{enumerate}
\end{algorithm}


\begin{remark}
Im Fall von Ausdrücken
mit wenigen Wurzeloperationen 
berechnet der {\tt ZVAA}-Vorzeichentest
nach \refalgorithm{al:Vorzeichentest}
zwei nennenswerte zusätzliche Operationen
und ist deshalb etwas langsamer als {\tt BFMSS[2]}.
Bei wachsender Anzahl an unterschiedlichen Wurzeloperationen 
nähern sich beide im Fall $\xi_E\ne 0$ aneinander an,
und im Fall $\xi_E=0$ wird {\tt ZVAA} drastisch schneller.
Es empfielt sich, unter entsprechenden Umständen {\tt ZVAA} einzusetzen.
\end{remark}
