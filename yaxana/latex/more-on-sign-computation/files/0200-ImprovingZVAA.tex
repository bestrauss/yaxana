%
% LaTeX
%
\pagebreak
\section{The precision of an expression}

Let $E=E_0-E_1$ be an expression 
and $N$ be an unknown number of binary places of two approximations
of the values $\xi_{E_0}=\val E_0$ and $\xi_{E_1}=\val E_1$, 
that is sufficient to read the signum $\sgn\xi_E$
from the exact difference of the two approximations.
We consider four distinct cases:
\begin{enumerate}
\item The abstract syntax trees (AST) of $E_0$ and $E_1$ are identical. 
	  Structure, operators, and operands are the same.
\item The ASTs of $E_0$ and $E_1$ have the same
	  structure and operators, but the operands are not all the same.
\item The ASTs of $E_0$ and $E_1$ have a different
	  structure and operators, but depending on the choice of operands
	  the values of both expressions may be identical.
\item The ASTs of $E_0$ and $E_1$ have a different
	  structure and operators, and whatever the choice of operands
	  the values of both expressions always differ.
\end{enumerate}
We can handle the cases as follows:
\begin{enumerate}
\item This is tested in advance, if detected, the signum is $\sgn\xi_E=0$.
\item We determine two numbers of binary places for $E_0$ and $E_1$
      as described below, and choose $N$ as the maximum of them.
\item TODO
\item TODO
\end{enumerate}


\section{The second case}

We're looking for a sufficient precision for an expression $E$ 
that makes sure that our approximation of the value of $E$
is different from all changed expressions $E'$, which have
the same structure and operands as $E$ but whose operands
bitpattern may differ within the given significant bits.


\section{The third case}

A given precision $N$ which suffices for the second case,
suffices for the third case, too.




\pagebreak
\section{Using ZVAA}

Let $E=E_0-E_1$ be a divisionfree expression with integer operands.
Let $P_E$ be the structural polynomial of $E$
\begin{equation}
  P_E = \sum_{k=0}^{N} c_k z^k = \prod_{k=1}^N (z - z_k)
  ,\qquad
  z\in\C, c_k\in\Z, z_k\in\A
\end{equation}
Then $|c_0|\in\N$, consequently $|c_0|\ge 1$, and we have
\begin{equation}
  \prod_{k=1}^N |z_k| = |c_0| \ge 1.
\end{equation}
Let $E'$ be 
\begin{equation}
  E' = \frac{E_0-E_1}{|E_0|+|E_1|}
  ,\qquad
  u\in\Q, u\ge\max_k|z_k|.
\end{equation}
Then
\begin{equation}
  P_E=P_{E_0-E_1}=P_{|E_0|+|E_1|}
\end{equation}
and the roots $z'_{E,k}$ of $P_{E'}$ 
either lie on the unit circle $|z'_{E,k}|=1$ of the gaussian $z$-plane,
or pairwise products $|z'_{E,k}z'_{E,j}|=1$ do.
Consequently, given an upper bound $u_{E'}\ge\max_k|z'_{E,k}|$,
the reciprocal value is a lower bound
\begin{equation}
  \frac{1}{u_{E'}}\le\min_k|z'_{E,k}|.
\end{equation}
The problem is, how to calculate $u_{E'}$.
We'd need an upper bound for the roots of the nominator, which is easy,
and a lower bound for the for the roots of the denominator, 
which is harder.

Given the {\em triangle inequality}, we have $|\val E'|<1$.



\pagebreak
\section{Improving ZVAA}

The ZVAA method uses the test-expression
\begin{equation}
  E' = \frac{E_0 - E_1}{|E_0| + |E_1|}
\end{equation}
to compute the sign of the value of the expression
\begin{equation}
  E = E_0 - E_1
  ,\qquad
  \val E_0\ne 0, \val E_1\ne 0.
\end{equation}
The signs of the values of the expressions
$E$ and $E'$ are obviously equal.
As has been shown in \cite{ZVAA}
The ZVAA method provides a root separation bound
for expressions of type $E'$, 
which dominates the bounds of other methods.
But given expressions of the type $E$,
the ZVAA algorithm treats expressions of the type $E'$
with a few more operations.
The ZVAA algorithm computes a bound for $E'$
and then computes approximations for $E$.

Let 
\begin{equation}
  d = \val(|E_0| + |E_1|).
\end{equation}
then a zero separation bound 
for the value $\xi_E'=\val E'$ where
\begin{equation}
  E' = \frac{E_0 - E_1}{|E_0| + |E_1|}
\end{equation}
also is a zero separation bound 
for the value $\xi_E''=\val E''$ where
\begin{equation}
  E'' = \frac{E_0 - E_1}{d}.
\end{equation}
Now let $1 \le d \le 2^{M_+}$,
then 
\begin{equation}
  E''' = \frac{E_0 - E_1}{2^{M_+}}.
\end{equation}
and
\begin{equation}
  \val E''' \le \val E''
\end{equation}
$\sep E'''$ can be used for $E''$.



\section{BFMSS}

Let $E_N$ and $E_D$ be division free expressions,
and
\begin{equation}
  \frac{1}{u_{E_N}^{D-1}} \le |\xi_{E_N}| \le u_{E_N}
  \quad\land\quad
  \frac{1}{u_{E_D}^{D-1}} \le |\xi_{E_D}| \le u_{E_D}
\end{equation}
then
\begin{equation}
  \frac{1}{u_{E_D}} \le \frac{1}{|\xi_{E_D}|} \le u_{E_D}^{D-1},
\end{equation}
and finally
\begin{equation}
  \frac{1}{u_{E_D}u_{E_N}^{D-1}} \le \frac{|\xi_{E_N}|}{|\xi_{E_D}|} \le u_{E_N}u_{E_D}^{D-1}.
\end{equation}
Now, in contrast, let $E_N$ and $E_D$ be division free expressions,
and
\begin{equation}
  \frac{1}{u_{E_N}^{D-1}} \le |\xi_{E_N}| \le u_{E_N}
  \quad\land\quad
  \floor{|\xi_{E_D}|} \le |\xi_{E_D}| \le \ceil{|\xi_{E_D}|}
\end{equation}
then
\begin{equation}
  \frac{1}{\ceil{|\xi_{E_D}|}} \le \frac{1}{|\xi_{E_D}|} \le \frac{1}{\floor{|\xi_{E_D}|}},
\end{equation}
and finally
\begin{equation}
  \frac{1}{\ceil{|\xi_{E_D}|}u_{E_N}^{D-1}} \le \frac{|\xi_{E_N}|}{|\xi_{E_D}|} \le \frac{u_{E_N}}{\floor{|\xi_{E_D}|}}.
\end{equation}


\section{ZVAA}

Mein Problem besteht irgendwie darin, 
daß ZVAA nur für divisonsfreie Ausdrücke gilt.
Dort ist das Produkt der Beträge der Nullstellen nicht $\in\N$
sondern $\in\Q$.
Das Produkt der Beträge der Nullstellen ist für 
wohlgeratene Ausdrücke aber immer $1$, 
egal ob divisonsfrei oder nicht.

BMFSS berechnet eine obere Nullstellenschranke für den Nenner,
und verwendet dann den Kehrwert davon mit Exponent versehen
als untere Nullstellenschranke für den Nenner.
Beim Nenner besteht aber nun kein Risiko, 
daß er verschwinden könnte. 
Wir können seinen Wert bis auf 52 Stellen genau ausrechnen,
und dann ohne Aufwand obere und untere Schranke angeben.

Unser Ausdruck kann nicht gleichzeitig divisionsfrei und
wohlgeraten sein.

Let 
\begin{equation}
  E = E_0 - E_1
\end{equation}
be an expression.
Then
\begin{equation}
  E' = \frac{E_0 - E_1}{|E_0| + |E_1|}
\end{equation}
is well worked.
Nominator and denominator expressions 
have one and the same structural polynomial.
A lower root bound for $|E_0| + |E_1|$ is
a lower root bound for $E_0 - E_1$.
\begin{equation}
  E' = \frac{E_0}{|E_0| + |E_1|}-\frac{E_1}{|E_0| + |E_1|}
\end{equation}
\begin{equation}
  E'' = 1-\frac{\frac{E_1}{|E_0| + |E_1|}}{\frac{E_0}{|E_0| + |E_1|}}
      = 1-\frac{E_1}{E_0}
\end{equation}



Let 
\begin{equation}
  E = E_0 - E_1 = \frac{A_0}{B_0}-\frac{A_1}{B_1}
\end{equation}
be an expression, 
where the $A$s and $B$s are division free expressions.
Then 
\begin{equation}
  A_0B_1 - A_1B_0
\end{equation}
is a test expression for $E$.
Let furthermore
\begin{equation}
  E' = \frac{A_0B_1 - A_1B_0}{|A_0B_1| + |A_1B_0|},
\end{equation}
then $E'$ is a well worked test expression for $E$.


Für einen allgemeinen, i.a. nicht divisionsfreien Ausdruck $E$
ist das Produkt der Nullstellen rational. Für unseren Testausdruck
ist das Produkt der Nullstellen aber immer $1$ und daher natürlich.
Wir können also einfach eine obere Nullstellenschranke berechnen.

Let
\begin{equation}
  E = E_0 - E_1.
\end{equation}
Let
\begin{equation}
  E' = 1 - \frac{E_1}{E_0}.
\end{equation}
Then $\sgn E = \sgn E'$.
\begin{equation}
  E'' = \frac{1 - \frac{E_1}{E_0}}{1 + \left|\frac{E_1}{E_0}\right|}.
\end{equation}
Then $\sgn E = \sgn E''$.






\section{Stash}


\noindent Let $E$ be a division free expression $E=E_0-E_1$,
then
\begin{equation}
  \frac{1}{u_{E_N}^{D-1}} \le |\xi_{E_N}| \le u_{E_N}
  \quad\land\quad
  \frac{1}{u_{E_D}^{D-1}} \le |\xi_{E_D}| \le u_{E_D}
\end{equation}





\noindent Let $E'$ be a well worked test expression for the 
division free expression $E$
and $u_{E'}$ an upper root bound\footnote{
upper bound of the absolute values of the roots of the minimal polygonial of}
of $E'$, then
\begin{equation}
  \frac{1}{u_{E'}} \le |\xi_{E'}| \le u_{E'}
\end{equation}


\begin{equation}
  E' = \frac{E_0-E_1}{|E_0|+|E_1|}
\end{equation}
\begin{equation}
  \val E' =   \frac{\val(E_0-E_1)}{\val(|E_0|+|E_1|)}
          \ge \frac{\val(E_0-E_1)}{u_{|E_0|+|E_1|}}
          =   \frac{\val(E_0-E_1)}{u_{E_0-E_1}}
\end{equation}
\begin{equation}
  \val(E_0-E_1) \le \val(E') u_{E_0-E_1}
\end{equation}
