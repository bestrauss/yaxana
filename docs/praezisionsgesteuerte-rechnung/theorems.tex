%
% LaTeX
%

% Theoreme
% https://tools.ietf.org/doc/texlive-doc/latex/ntheorem/ntheorem.pdf
\usepackage[amsmath,thmmarks,framed]{ntheorem}
\usepackage{framed} 
%\usepackage{amsthm}

%
% Definitionen, Theoreme, etc.
%
\makeatletter
\def\newColoredTheorem#1#2{%
\theoremprework{\vskip\theoremframepreskipamount
\renewcommand*\FrameCommand{%
  {\color{#2}\vrule width 3pt \hspace{15pt}}}
  \framed}%
\theorempostwork{\endframed\vskip\theoremframepostskipamount}%
\newtheorem@i{#1}%
}
\makeatother

\makeatletter
\def\newColoredFSTheorem#1#2{%
\theoremprework{\vskip\theoremframepreskipamount
\renewcommand*\FrameCommand{%
  {\color{#2}\vrule width 3pt \hspace{15pt}}}
  \framed\begin{samepage}\footnotesize}%
\theorempostwork{\end{samepage}\endframed\vskip\theoremframepostskipamount}%
\newtheorem@i{#1}%
}
\makeatother

\definecolor{defiColor}{rgb}{0.3,0.5,1.0}
\definecolor{examColor}{rgb}{0.3,0.5,1.0}
\definecolor{theoColor}{rgb}{1.0,0.3,0.5}
\definecolor{lemmColor}{rgb}{0.0,0.6,0.0}
\definecolor{corrColor}{rgb}{0.0,0.6,0.0}
\definecolor{remaColor}{rgb}{1.0,0.7,0.0}
\definecolor{algoColor}{rgb}{0.0,1.0,0.7}
\definecolor{conjColor}{rgb}{1.0,0.0,0.0}
\definecolor{noteColor}{rgb}{1.0,0.9,0.3}
\definecolor{figuColor}{rgb}{0.9,0.9,0.9}

\theoremstyle{plain}
\newColoredTheorem{definition}{defiColor}{Definition}[section]
\newColoredTheorem{theorem}{theoColor}{Satz}[section]
\newColoredTheorem{lemma}{lemmColor}{Lemma}[section]
\newColoredTheorem{corollary}{corrColor}{Korollar}[section]
\newColoredTheorem{algorithm}{algoColor}{Algorithmus}[section]
\newColoredTheorem{conjecture}{conjColor}{Vermutung}[section]
\newColoredTheorem{example}{examColor}{Beispiel}[section]
\newColoredTheorem{Figure}{figuColor}{Abbildung}[section]

\theorembodyfont{\normalfont}
\theoremsymbol{$\square$}
\newtheorem{proof}{Beweis}[section]
\theoremsymbol{}
\newColoredTheorem{remark}{remaColor}{Anmerkung}[section]

\theoremstyle{nonumberplain}
\theorembodyfont{\normalfont}
\newColoredFSTheorem{note}{noteColor}{Hinweis:}

\theoremstyle{nonumberplain}
\newtheorem{dummy}{}
\newcommand{\pagebreakhint}{\begin{dummy}\end{dummy}}

\newcommand{\refchapter}[1]{Kapitel~\ref{#1}       auf Seite~\pageref{#1}}
\newcommand{\refsection}[1]{Abschnitt~\ref{#1}     auf Seite~\pageref{#1}}
\newcommand{\refequation}[1]{Gleichung~\ref{#1}    auf Seite~\pageref{#1}}
\newcommand{\refappendix}[1]{Anhang~\ref{#1}       auf Seite~\pageref{#1}}
\newcommand{\reffigure}[1]{Bild~\ref{#1}           auf Seite~\pageref{#1}}
\newcommand{\reftable}[1]{Tabelle~\ref{#1}         auf Seite~\pageref{#1}}
\newcommand{\refdefinition}[1]{Definition~\ref{#1} auf Seite~\pageref{#1}}
\newcommand{\reftheorem}[1]{Satz~\ref{#1}          auf Seite~\pageref{#1}}
\newcommand{\reflemma}[1]{Lemma~\ref{#1}           auf Seite~\pageref{#1}}
\newcommand{\refremark}[1]{Anmerkung~\ref{#1}      auf Seite~\pageref{#1}}
\newcommand{\refalgorithm}[1]{Algorithmus~\ref{#1} auf Seite~\pageref{#1}}
\newcommand{\refexample}[1]{Beispiel~\ref{#1}      auf Seite~\pageref{#1}}
\newcommand{\refconjecture}[1]{Vermutung~\ref{#1}  auf Seite~\pageref{#1}}
%
% EOF
%