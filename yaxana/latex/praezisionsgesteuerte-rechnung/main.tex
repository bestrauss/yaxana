%
% LaTeX
%
\documentclass[11pt,fleqn,german]{article}
\usepackage[utf8]{inputenc}
\usepackage[OT2,T1]{fontenc}
\usepackage[german]{babel}
\usepackage{cite}
\usepackage{fancyhdr}
\usepackage{graphicx}
\usepackage{curves}
\usepackage{tkz-graph}
\usepackage{color}
%\usepackage[table]{xcolor}
\usepackage{amsmath}
\usepackage{amsfonts}
\usepackage{amssymb}
\usepackage{esdiff}          % Ableitungen
\usepackage{hyperref}
\usepackage{enumitem}
  \setlist{topsep=0.4em,parsep=0em,itemsep=0.2em}
\usepackage{changepage} % adjustwidth

% https://math.uoregon.edu/wp-content/uploads/2014/12/compsymb-1qyb3zd.pdf

\newcommand{\Author}{Burkhard E. Strauß}
\newcommand{\Date}{Mai 2017}
\newcommand{\Rev}{Rev 1.0}
\newcommand{\thetitle}{Präzisionsgesteuerte Rechnung}
\title{\thetitle}


% http://de.wikibooks.org/wiki/LaTeX-W%C3%B6rterbuch:_fancyhdr
\pagestyle{fancy}
\fancyhf{} %alle Kopf- und Fußzeilenfelder bereinigen
\fancyhead[EL]{\thepage}
\fancyhead[EC]{}
\fancyhead[ER]{\thetitle}
\fancyhead[OL]{\thetitle}
\fancyhead[OC]{}
\fancyhead[OR]{\thepage}
%\fancyfoot[ER]{{\small\Site}}
%\fancyfoot[OL]{{\small\Site}}
%\fancyfoot[EC]{{\small\Revision}}
%\fancyfoot[OC]{{\small\Revision}}
%\fancyfoot[EL]{{\small\Copyright}}
%\fancyfoot[OR]{{\small\Copyright}}
\renewcommand{\headrulewidth}{0.5pt} % obere Trennlinie
%\renewcommand{\footrulewidth}{0.5pt} % untere Trennlinie
\setlength{\headheight}{15pt}

%
% Definitionen
%
\definecolor{background}{rgb}{1.0,1.0,1.0}
\pagecolor{background}
\newcommand{\comma}{\;,\;\;}
\newcommand{\warning}[1]{\textbf{!!! #1 !!!}}
\newcommand{\skipline}{\vskip 1em}
%\newcommand{\skipline}{\begin{color}{background}.\end{color}}
\newcommand{\eqdef}{\mathrel{\mathop:}=}
\newcommand{\equivalent}{\;\;\;\Leftrightarrow\;\;\;}
\newcommand{\map}[1]{%
\setlength{\unitlength}{1ex}%
\def\bullet{\circle{1.0}}%
\;%
\begin{picture}(7, 1)
  \put( 0.5,  0.5){\bullet}
  \put( 6.5,  0.5){\bullet}
  \put( 1.0,  0.5){\line( 1, 0){5.0}}
  \put( 0.0,  1.3){\makebox[7ex]{\footnotesize $#1$}}
\end{picture}\;\;}

\hyphenation{
  hin-rei-chend hin-rei-chen-de 
  Dar-stel-lung-en 
  Null-tren-nungs-schran-ke
  Ter-mi-nal-funk-tion
  In-ter-vall-arith-metik
}

%
%  symbols.tex
%

\newcommand*{\defeq}{\mathrel{\vcenter{\baselineskip0.5ex \lineskiplimit0pt
                     \hbox{\scriptsize.}\hbox{\scriptsize.}}}%
                     =}

% Schriftgrößen
\def\mD{\displaystyle}
\def\mT{\textstyle}
\def\mS{\scriptstyle}
\def\mSS{\scriptscriptstyle}


% Zahlenmengen
\newcommand{\N}{\mathbb{N}}
\newcommand{\Z}{\mathbb{Z}}
\newcommand{\Q}{\mathbb{Q}}
\newcommand{\A}{\mathbb{A}}
\newcommand{\R}{\mathbb{R}}
\renewcommand{\C}{\mathbb{C}}

% Impulsfolge
\DeclareSymbolFont{cyrletters}{OT2}{wncyr}{m}{n}
\DeclareMathSymbol{\Sha}{\mathalpha}{cyrletters}{"58}

% Allgemein
\newcommand{\idx}[3]{_{#1,#2}(#3)}
\newcommand{\idxQ}[3]{_{#1,#2}^{\,2}(#3)}
\newcommand{\Skt}{\idx{S}{k}{t}}
\newcommand{\SKt}{\idx{S}{K}{t}}
\newcommand{\SktQ}{\idxQ{S}{k}{t}}
\newcommand{\sgn}{\operatorname{sgn}}
\newcommand{\w}{{\rm w}}
\newcommand{\ejo}{e^{j\omega}}
\newcommand{\p}{\Sha}
\newcommand{\pM}{\p_{{\mSS M}}}
\newcommand{\pN}{\p_{{\mSS N}}}
\newcommand{\M}[1]{{\bf #1}}
\newcommand{\Hpr}{H_{{\rm pr}}}
\newcommand{\hpr}{h_{{\rm pr}}}
\newcommand{\poly}[1]{\hat{#1}} 

\newcommand{\mth}{$M^{{\rm th}}$ }
\newcommand{\mthband}{$M^{{\rm th}}$-band }
\newcommand{\nth}{$N^{{\rm th}}$ }
\newcommand{\nthband}{$N^{{\rm th}}$-band }
\renewcommand{\th}[1]{${#1}^{{\rm th}}$}

\newcommand{\cN}[1]{\mathop{\rm c}_{N,{#1}}\nolimits}
\newcommand{\CN}[1]{\mathop{\rm C}_{N,{#1}}\nolimits}

\def\ld{\mathop{\rm ld}\nolimits}
\def\val{\mathop{\rm val}\nolimits}
\def\sort{\mathop{\rm sort}\nolimits}
\def\odd{\mathop{\rm Odd}\nolimits}
\def\even{\mathop{\rm Even}\nolimits}
\def\Im{\mathop{\rm Im}\nolimits}
\def\Re{\mathop{\rm Re}\nolimits}


\def\pc{\mathop{\rm pc}\nolimits}
\def\PC{\mathop{\rm PC}\nolimits}
\def\pd{\mathop{\rm pd}\nolimits}
\def\PD{\mathop{\rm PD}\nolimits}

% Allpässe
\def\AP{\mathop{\rm AP}\nolimits}
\def\ap{\mathop{\rm ap}\nolimits}
\def\BP{\mathop{\widehat{{\rm AP}}}\nolimits}
\def\bp{\mathop{\widehat{{\rm ap}}}\nolimits}
\def\NAP{\mathop{\widetilde{{\rm AP}}}\nolimits}
\def\Nap{\mathop{\widetilde{{\rm ap}}}\nolimits}
\def\Nphi{\mathop{\widetilde{\varphi}}\nolimits}

\newcommand{\cpx}[1]{\underline{#1}}
\newcommand{\expj}[1]{e^{j{#1}}}



%
%  Transformationssymbole
%

\def\Bull{\circle*{1.5}}
\def\Circ{\circle{1.5}}

\newcommand{\vxtoy}[4]{\lower0.6em\hbox{%         vertikal
\unitlength 0.237em
\begin{picture}(4,12)
\put(1,1){#4}
\put(1,11){#1}
\put(1,1.75){\line(0,1){8.5}}
\put(3,3){\makebox(0,0){$\mS #3$}}
\put(3,9){\makebox(0,0){$\mS #2$}}
\put(3,6){\makebox(0,0){$\mS\updownarrow$}}
\end{picture}}}

\newcommand{\hxtoy}[4]{\;\raise0.06em\hbox{%        horizontal
\unitlength 0.237em
\begin{picture}(11,4)
\put(1,1){#1}
\put(10,1){#4}
\put(1.75,1){\line(1,0){7.5}}
\put(3.2,2.75){\makebox(0,0){$\mS #2$}}
\put(7.8,2.75){\makebox(0,0){$\mS #3$}}
\put(5.5,2.75){\makebox(0,0){$\mS\leftrightarrow$}}
\end{picture}}\;}

\def\voton{\vxtoy{\Bull}{\omega}{n}{\Circ}}      % \omega <-> n
\def\vntoo{\vxtoy{\Circ}{n}{\omega}{\Bull}}
\def\oton{\hxtoy{\Bull}{\omega}{n}{\Circ}}
\def\ntoo{\hxtoy{\Circ}{n}{\omega}{\Bull}}
\def\vzton{\vxtoy{\Bull}{z}{n}{\Circ}}           % z <-> n
\def\vntoz{\vxtoy{\Circ}{n}{z}{\Bull}}
\def\zton{\hxtoy{\Bull}{z}{n}{\Circ}}
\def\ntoz{\hxtoy{\Circ}{n}{z}{\Bull}}
\def\vnuton{\vxtoy{\Circ}{\nu}{n}{\Circ}}        % \nu <-> n
\def\vntonu{\vxtoy{\Circ}{n}{\nu}{\Circ}}
\def\nuton{\hxtoy{\Circ}{\nu}{n}{\Circ}}
\def\ntonu{\hxtoy{\Circ}{n}{\nu}{\Circ}}
\def\vOtoo{\vxtoy{\Bull}{\Omega}{\omega}{\Bull}} % \omega <-> \Omega
\def\votoO{\vxtoy{\Bull}{\omega}{\Omega}{\Bull}}
\def\Otoo{\hxtoy{\Bull}{\Omega}{\omega}{\Bull}}
\def\otoO{\hxtoy{\Bull}{\omega}{\Omega}{\Bull}}
\def\vOtonu{\vxtoy{\Bull}{\Omega}{\nu}{\Circ}}   % \Omega <-> \nu
\def\vnutoO{\vxtoy{\Circ}{\nu}{\Omega}{\Bull}}
\def\Otonu{\hxtoy{\Bull}{\Omega}{\nu}{\Circ}}
\def\nutoO{\hxtoy{\Circ}{\nu}{\Omega}{\Bull}}

%
%  EOF
%

%
% LaTeX
%

% Theoreme
% https://tools.ietf.org/doc/texlive-doc/latex/ntheorem/ntheorem.pdf
\usepackage[amsmath,thmmarks,framed]{ntheorem}
\usepackage{framed} 
%\usepackage{amsthm}

%
% Definitionen, Theoreme, etc.
%
\makeatletter
\def\newColoredTheorem#1#2{%
\theoremprework{\vskip\theoremframepreskipamount
\renewcommand*\FrameCommand{%
  {\color{#2}\vrule width 3pt \hspace{15pt}}}
  \framed}%
\theorempostwork{\endframed\vskip\theoremframepostskipamount}%
\newtheorem@i{#1}%
}
\makeatother

\makeatletter
\def\newColoredFSTheorem#1#2{%
\theoremprework{\vskip\theoremframepreskipamount
\renewcommand*\FrameCommand{%
  {\color{#2}\vrule width 3pt \hspace{15pt}}}
  \framed\begin{samepage}\footnotesize}%
\theorempostwork{\end{samepage}\endframed\vskip\theoremframepostskipamount}%
\newtheorem@i{#1}%
}
\makeatother

\definecolor{defiColor}{rgb}{0.3,0.5,1.0}
\definecolor{examColor}{rgb}{0.3,0.5,1.0}
\definecolor{theoColor}{rgb}{1.0,0.3,0.5}
\definecolor{lemmColor}{rgb}{0.0,0.6,0.0}
\definecolor{corrColor}{rgb}{0.0,0.6,0.0}
\definecolor{remaColor}{rgb}{1.0,0.7,0.0}
\definecolor{algoColor}{rgb}{0.0,1.0,0.7}
\definecolor{conjColor}{rgb}{1.0,0.0,0.0}
\definecolor{noteColor}{rgb}{1.0,0.9,0.3}
\definecolor{figuColor}{rgb}{0.9,0.9,0.9}

\theoremstyle{plain}
\newColoredTheorem{definition}{defiColor}{Definition}[section]
\newColoredTheorem{theorem}{theoColor}{Satz}[section]
\newColoredTheorem{lemma}{lemmColor}{Lemma}[section]
\newColoredTheorem{corollary}{corrColor}{Korollar}[section]
\newColoredTheorem{algorithm}{algoColor}{Algorithmus}[section]
\newColoredTheorem{conjecture}{conjColor}{Vermutung}[section]
\newColoredTheorem{example}{examColor}{Beispiel}[section]
\newColoredTheorem{Figure}{figuColor}{Abbildung}[section]

\theorembodyfont{\normalfont}
\theoremsymbol{$\square$}
\newtheorem{proof}{Beweis}[section]
\theoremsymbol{}
\newColoredTheorem{remark}{remaColor}{Anmerkung}[section]

\theoremstyle{nonumberplain}
\theorembodyfont{\normalfont}
\newColoredFSTheorem{note}{noteColor}{Hinweis:}

\theoremstyle{nonumberplain}
\newtheorem{dummy}{}
\newcommand{\pagebreakhint}{\begin{dummy}\end{dummy}}

\newcommand{\refchapter}[1]{Kapitel~\ref{#1}       auf Seite~\pageref{#1}}
\newcommand{\refsection}[1]{Abschnitt~\ref{#1}     auf Seite~\pageref{#1}}
\newcommand{\refequation}[1]{Gleichung~\ref{#1}    auf Seite~\pageref{#1}}
\newcommand{\refappendix}[1]{Anhang~\ref{#1}       auf Seite~\pageref{#1}}
\newcommand{\reffigure}[1]{Bild~\ref{#1}           auf Seite~\pageref{#1}}
\newcommand{\reftable}[1]{Tabelle~\ref{#1}         auf Seite~\pageref{#1}}
\newcommand{\refdefinition}[1]{Definition~\ref{#1} auf Seite~\pageref{#1}}
\newcommand{\reftheorem}[1]{Satz~\ref{#1}          auf Seite~\pageref{#1}}
\newcommand{\reflemma}[1]{Lemma~\ref{#1}           auf Seite~\pageref{#1}}
\newcommand{\refremark}[1]{Anmerkung~\ref{#1}      auf Seite~\pageref{#1}}
\newcommand{\refalgorithm}[1]{Algorithmus~\ref{#1} auf Seite~\pageref{#1}}
\newcommand{\refexample}[1]{Beispiel~\ref{#1}      auf Seite~\pageref{#1}}
\newcommand{\refconjecture}[1]{Vermutung~\ref{#1}  auf Seite~\pageref{#1}}
%
% EOF
%

%
% Dokument
%
\begin{document}

\author{\Author}
\date{\Date}
\maketitle
\vfill
\tableofcontents

%
% LaTeX
%
\pagebreak
\section{Addition}

\begin{equation*}
\begin{split}
  |\hat{z}-z| & =  |\hat{z}-(x+y)| \\
              & =  |\hat{z}-(\hat{x}+(x-\hat{x}))-(\hat{y}+(y-\hat{y}))| \\
              & =  |\hat{z}-(\hat{x}+\hat{y})-(x-\hat{x})-(y-\hat{y})| \\
              & =  |\hat{z}-(\hat{x}+\hat{y})+(\hat{x}-x)+(\hat{y}-y)| \\
              &\le |\hat{z}-(\hat{x}+\hat{y})|+|\hat{x}-x|+|\hat{y}-y| \\
              &\le 2^{-p_z}
\end{split}
\end{equation*}

\begin{equation*}
\begin{split}
  |\hat{z}-(\hat{x}+\hat{y})| & \le & \frac{2^{-p_z}}{2}
  & \Leftrightarrow &
  2^{-p_+}|\hat{z}| & \le & 2^{-(p_z+1)}
  & \Leftrightarrow &
  2^{-p_+} \le 2^{-(p_z+2+\ld|\hat{z}|)}
  \\
  |\hat{x}-x| & \le & \frac{2^{-p_z}}{4}
  & \Leftrightarrow &
  2^{-p_x} & \le & 2^{-(p_z+2)}
  &  &
  \\
  |\hat{y}-y| & \le & \frac{2^{-p_z}}{4}
  & \Leftrightarrow &
  2^{-p_y} & \le & 2^{-(p_z+2)}
  &  &
  \\
\end{split}
\end{equation*}

\begin{equation*}
  \ld|\hat{z}| \le \max(\ld|\hat{x}|,\ld|\hat{y}|)+1
\end{equation*}

\begin{algorithm}
\phantom{}
\begin{center}
\renewcommand*{\arraystretch}{1.3}
\begin{tabular}{lcl}
  \hline
  \multicolumn{3}{c}{Addition} \\
  \hline
   $p_x$ & $\defeq$ & $p_z+2$ \\
   $p_y$ & $\defeq$ & $p_z+2$ \\
   $p_+$ & $\defeq$ & $p_z+3+\max(\ld|\hat{x}|,\ld|\hat{y}|)$ \\
  \hline
\end{tabular}
\end{center}
\end{algorithm}


\section{Subtraktion}

{\em ---Herleitung analog zur Addition---}

\begin{algorithm}
\phantom{}
\begin{center}
\renewcommand*{\arraystretch}{1.3}
\begin{tabular}{lcl}
  \hline
  \multicolumn{3}{c}{Subtraktion} \\
  \hline
   $p_x$ & $\defeq$ & $p_z+2$ \\
   $p_y$ & $\defeq$ & $p_z+2$ \\
   $p_-$ & $\defeq$ & $p_z+3+\max(\ld|\hat{x}|,\ld|\hat{y}|)$ \\
  \hline
\end{tabular}
\end{center}
\end{algorithm}



\pagebreak
\section{Multiplikation}

\begin{equation*}
\begin{split}
  |\hat{z}-z| & =  |\hat{z}-xy| \\
              & =  |\hat{z}-(\hat{x}+(x-\hat{x}))(\hat{y}+(y-\hat{y}))| \\
              & =  |\hat{z}-\hat{x}\hat{y}-\hat{x}(y-\hat{y})-\hat{y}(x-\hat{x})-(x-\hat{x})(y-\hat{y})| \\
              & =  |\hat{z}-\hat{x}\hat{y}+\hat{x}(\hat{y}-y)+\hat{y}(\hat{x}-x)-(\hat{x}-x)(\hat{y}-y)| \\
              &\le |\hat{z}-\hat{x}\hat{y}|+|\hat{x}(\hat{y}-y)|+|\hat{y}(\hat{x}-x)|+|(\hat{x}-x)(\hat{y}-y)| \\
              &\le 2^{-p_z}
\end{split}
\end{equation*}

\begin{equation*}
\begin{split}
  |\hat{z}-\hat{x}\hat{y}| & \le & \frac{2^{-p_z}}{4}
  & \Leftrightarrow & 
  2^{-{p_\times}}|\hat{z}| & \le & 2^{-(p_z+2)}
  & \Leftrightarrow & 
  \cdots
  \\
  |\hat{x}(\hat{y}-y)| & \le & \frac{2^{-p_z}}{4}
  & \Leftrightarrow & 
  2^{-p_y}|\hat{x}|  & \le & 2^{-(p_z+2)}
  & \Leftrightarrow & 
  \cdots
  \\
  |\hat{y}(\hat{x}-x)| & \le & \frac{2^{-p_z}}{4}
  & \Leftrightarrow & 
  2^{-p_x}|\hat{y}|  & \le & 2^{-(p_z+2)}
  & \Leftrightarrow & 
  \cdots
  \\
  |(\hat{x}-x)(\hat{y}-y)| & \le & \frac{2^{-p_z}}{4}
  & \Leftrightarrow & 
  2^{-(p_x+p_y)}|  & \le & 2^{-(p_z+2)}
  &  & 
  \\
\end{split}
\end{equation*}

\begin{equation*}
\begin{split}
  \cdots
  & \Leftrightarrow & 
  2^{-{p_\times}} & \le & 2^{-(p_z+2+\ld|\hat{z}|)}
  \\
  \cdots
  & \Leftrightarrow & 
  2^{-p_y}  & \le & 2^{-(p_z+2+\ld|\hat{x}|)}
  \\
  \cdots
  & \Leftrightarrow & 
  2^{-p_x} & \le & 2^{-(p_z+2+\ld|\hat{y}|)}
  \\
\end{split}
\end{equation*}

\begin{equation*}
  \ld|\hat{z}| \le \ld|\hat{x}|+\ld|\hat{y}|+1
\end{equation*}

\begin{algorithm}
\phantom{}
\begin{center}
\renewcommand*{\arraystretch}{1.3}
\begin{tabular}{lcl}
  \hline
  \multicolumn{3}{c}{Multiplikation} \\
  \hline
   $p_x$      & $\defeq$ & $p_z+2+\ld|\hat{y}|$ \\
   $p_y$      & $\defeq$ & $p_z+2+\ld|\hat{x}|$ \\
   $p_\times$ & $\defeq$ & $p_z+3+\ld|\hat{x}|+\ld|\hat{y}|$ \\
  \hline
\end{tabular}
\end{center}
\end{algorithm}




\pagebreak
\section{Division}

\begin{equation*}
\begin{split}
  |\hat{z}-z| & =  |\hat{z}-\frac{x}{y}| 
                =  \left|\hat{z}-\frac{\hat{x}}{\hat{y}}
			    +  \frac{\hat{x}}{\hat{y}}-\frac{x}{y}\right| \\
              &\le \left|\hat{z}-\frac{\hat{x}}{\hat{y}}\right|
			    +  \left|\frac{\hat{x}}{\hat{y}}-\frac{x}{y}\right| \\
              & =  \left|\hat{z}-\frac{\hat{x}}{\hat{y}}\right|
			    +  \left|\frac{x\hat{y}-\hat{x}y}{y\hat{y}}\right| \\
              & =  \left|\hat{z}-\frac{\hat{x}}{\hat{y}}\right|
			    +  \frac{|x\hat{y}-\hat{x}\hat{y}-\hat{x}y+\hat{x}\hat{y}|}{|y\hat{y}|} \\
              & =  \left|\hat{z}-\frac{\hat{x}}{\hat{y}}\right|
			    +  \frac{|(x-\hat{x})\hat{y}-(y-\hat{y})\hat{x}|}{|y\hat{y}|} \\
              & =  \left|\hat{z}-\frac{\hat{x}}{\hat{y}}\right|
			    +  \frac{|(x-\hat{x})\hat{y}|}{|y\hat{y}|}
			    +  \frac{|(\hat{y}-y)\hat{x}|}{|y\hat{y}|} \\
              & =  \left|\hat{z}-\frac{\hat{x}}{\hat{y}}\right|
			    +  \frac{|x-\hat{x}|}{|y|}
			    +  \frac{|(y-\hat{y})\hat{x}|}{|y\hat{y}|} \\
              &\le 2^{-p_z}
\end{split}
\end{equation*}

\begin{equation*}
\begin{split}
 & \left|\hat{z}-\frac{\hat{x}}{\hat{y}}\right|&&\le\frac{2^{-p_z}}{2}
   \Leftrightarrow &
   2^{-p_\div}|\hat{z}|&\le 2^{-(p_z+1)}
   \Leftrightarrow &
   \cdots
   \\
 & \frac{|x-\hat{x}|}{|y|}&&\le\frac{2^{-p_z}}{4}
   \Leftrightarrow &
   2^{-(p_x+\ld|y|)}&\le 2^{-(p_z+2)}
   \Leftrightarrow &
   \cdots
   \\
 & \frac{|(y-\hat{y})\hat{x}|}{|y\hat{y}|}&&\le\frac{2^{-p_z}}{4}
   \Leftrightarrow &
   2^{-(p_y-\ld|x|+\ld|y\hat{y}|)}&\le 2^{-(p_z+2)}
   \Leftrightarrow &
   \cdots
   \\
\end{split}
\end{equation*}

\begin{equation*}
\begin{split}
   &
   \cdots
   \Leftrightarrow &
   p_\div  &\ge p_z+1+\ld|\hat{z}|
   \\
   &
   \cdots
   \Leftrightarrow &
   p_x  &\ge p_z+2-\ld|y|
   \\
   &
   \cdots
   \Leftrightarrow &
   p_y  &\ge p_z+2+\ld|x|-\ld|y\hat{y}|
   \\
\end{split}
\end{equation*}

\begin{equation*}
  \ld|\hat{z}| \le \ld|\hat{x}|-\ld|\hat{y}|+1
\end{equation*}

\begin{algorithm}
\phantom{}
\begin{center}
\renewcommand*{\arraystretch}{1.3}
\begin{tabular}{lcl}
  \hline
  \multicolumn{3}{c}{Division} \\
  \hline
   $p_x$    & $\defeq$ & $p_z+2-\ld|y|$ \\
   $p_y$    & $\defeq$ & $p_z+2+\ld|x|-\ld|y\hat{y}|$ \\
   $p_\div$ & $\defeq$ & $p_z+2+\ld|\hat{x}|-\ld|\hat{y}|$ \\
  \hline
\end{tabular}
\end{center}
\end{algorithm}




%\pagebreak
\section{$n$-te Potenz}

\begin{equation*}
\begin{split}
  |\hat{z}-z| & =  |\hat{z}-x^n| \\
              & =  |\hat{z}-(\hat{x}+(x-\hat{x}))^n| \\
              & =  \left|\hat{z}-\sum_{k=0}^n\binom{n}{k}\hat{x}^{n-k}(x-\hat{x})^k\right| \\
              & =  \left|\hat{z}-\hat{x}^n-\sum_{k=1}^n\binom{n}{k}\hat{x}^{n-k}(x-\hat{x})^k\right| \\
              & =  \left|\hat{z}-\hat{x}^n+\sum_{k=1}^n\binom{n}{k}\hat{x}^{n-k}(\hat{x}-x)^k\right| \\
              &\le |\hat{z}-\hat{x}^n|+\sum_{k=1}^n\binom{n}{k}\left|\hat{x}^{n-k}(\hat{x}-x)^k\right| \\
              & =  |\hat{z}-\hat{x}^n|+\sum_{k=1}^n\binom{n}{k}|\hat{x}|^{n-k}|\hat{x}-x|^k \\
              &\le 2^{-p_z}
\end{split}
\end{equation*}

\begin{scriptsize}
\begin{equation*}
\begin{split}
  |\hat{z}-\hat{x}^n| & \le & \frac{2^{-p_z}}{n}
  & \Leftrightarrow & 
  2^{-{p_{(\phantom{\cdot})^n}}}|\hat{z}| & \le & 2^{-(p_z+\ld n)}
  & \Leftrightarrow & 
  \cdots
  \\
  \binom{n}{k}|\hat{x}|^{n-k}|\hat{x}-x|^k & \le & \frac{2^{-p_z}}{n}
  &  & &  & &  & 
  \cdots
  \\
  |\hat{x}|^{n-1}|\hat{x}-x| & \le & \frac{2^{-p_z}}{n}
  & \Leftrightarrow & 
  2^{-p_x}|\hat{x}|^{n-1}  & \le & 2^{-(p_z+\ld n)}
  & \Leftrightarrow & 
  \cdots
  \\
  \binom{n}{2}|\hat{x}|^{n-2}|\hat{x}-x|^2 & \le & \frac{2^{-p_z}}{n}
  & \Leftrightarrow & 
  \binom{n}{2} 2^{-2p_x}|\hat{x}|^{n-2}  & \le & 2^{-(p_z+\ld n)}
  & \Leftrightarrow & 
  \cdots
  \\
  \cdots
  \\
  \binom{n}{n-1}|\hat{x}||\hat{x}-x|^{n-1} & \le & \frac{2^{-p_z}}{n}
  & \Leftrightarrow & 
  \binom{n}{n-1}2^{-(n-1)p_x}|\hat{x}|  & \le & 2^{-(p_z+\ld n)}
  & \Leftrightarrow & 
  \cdots
  \\
  |\hat{x}-x|^n & \le & \frac{2^{-p_z}}{n}
  & \Leftrightarrow & 
  2^{-n p_x} & \le & 2^{-(p_z+\ld n)}
  & \Leftrightarrow & 
  \cdots
  \\
\end{split}
\end{equation*}
\end{scriptsize}

\begin{equation*}
\begin{split}
  \cdots
  & \Leftrightarrow & 
  2^{-p_{(\phantom{\cdot})^n}} & \le & 2^{-(p_z+\ld n+\ld|\hat{z}|)}
  \\
  \cdots
  & \Leftrightarrow & 
  2^{-p_x}  & \le & 2^{-(p_z+\ld n+(n-1)\ld|\hat{x}|)}
  \\
  \cdots
  & \Leftrightarrow & 
  2^{-2p_x}  & \le & 2^{-(p_z+\ld n+\ld\binom{n}{2}+(n-2)\ld|\hat{x}|)}
  \\
  & \cdots
  \\
  \cdots
  & \Leftrightarrow & 
  2^{-(n-1)p_x}  & \le & 2^{-(p_z+\ld n+\ld\binom{n}{n-1}+\ld|\hat{x}|)}
  \\
  \cdots
  & \Leftrightarrow & 
  2^{-n p_x}  & \le & 2^{-(p_z+\ld n)}
  \\
\end{split}
\end{equation*}

\begin{equation*}
\begin{split}
  \cdots
  & \Leftrightarrow & 
  p_{(\phantom{\cdot})^n} & \ge p_z+\ld n+\ld|\hat{z}|
  \\
  \cdots
  & \Leftrightarrow & 
  p_x  & \ge p_z+\ld n+(n-1)\ld|\hat{x}|
  \\
  \cdots
  & \Leftrightarrow & 
  p_x  & \ge \frac{1}{2}\left(p_z+\ld n+\ld\binom{n}{2}+(n-2)\ld|\hat{x}|\right)
  \\
  & \cdots
  \\
  \cdots
  & \Leftrightarrow & 
  p_x  & \ge \frac{1}{n-1}\left(p_z+\ld n+\ld\binom{n}{n-1}+\ld|\hat{x}|\right)
  \\
  \cdots
  & \Leftrightarrow & 
  p_x  & \ge \frac{1}{n}\left(p_z+\ld n\right)
  \\
\end{split}
\end{equation*}

\begin{equation*}
  \ld|\hat{z}| \le \ld|\hat{x}|+\ld|\hat{y}|+1 % TODO, auch Tabelle
\end{equation*}

\begin{algorithm}
\phantom{}
\begin{center}
\renewcommand*{\arraystretch}{1.3}
\begin{tabular}{lcl}
  \hline
  \multicolumn{3}{c}{$n$-te Potenz} \\
  \hline
   $p_x$      & $\defeq$ & $p_z+\ld n+\ld\binom{n}{n\div2}+(n-1)\ld|\hat{x}|$ \\
   $p_{(\phantom{\cdot})^n}$ & $\defeq$ & $p_z+\ld n+\ld|\hat{z}|$ \\
  \hline
\end{tabular}
\end{center}
\end{algorithm}






\pagebreak
\section{Quadratwurzel}

\newcommand{\op}{{\sqrt[2]{\phantom{\cdot}}}}

\begin{equation*}
\begin{split}
  |\hat{z}-z| & =  |\hat{z}-\sqrt{x}| \\
              & =  |\hat{z}-\sqrt{\hat{x}}+\sqrt{\hat{x}}-\sqrt{x}| \\
              &\le |\hat{z}-\sqrt{\hat{x}}|+|\sqrt{\hat{x}}-\sqrt{x}| \\
              & =  |\hat{z}-\sqrt{\hat{x}}|
			       +\left|\left(\sqrt{\hat{x}}-\sqrt{x}\right)
				    \frac{\sqrt{\hat{x}}+\sqrt{x}}{\sqrt{\hat{x}}+\sqrt{x}}\right| \\
              & =  |\hat{z}-\sqrt{\hat{x}}|+\frac{|\hat{x}-x|}{\sqrt{\hat{x}}+\sqrt{x}} \\
              &\le |\hat{z}-\sqrt{\hat{x}}|+\frac{|\hat{x}-x|}{\sqrt{\hat{x}}} \\
              &\le \frac{2^{-p_z}}{2}+\frac{2^{-p_z}}{2}=2^{-p_z}
\end{split}
\end{equation*}

\begin{equation*}
  |\hat{z}-\sqrt{\hat{x}}|           \le \frac{2^{-p_z}}{2}
  \quad,\quad
  \frac{|\hat{x}-x|}{\sqrt{\hat{x}}} \le \frac{2^{-p_z}}{2}
\end{equation*}

\begin{equation*}
\begin{split}
  |\hat{z}-\sqrt{\hat{x}}| & = 2^{-p_\op}|\hat{z}| \le \frac{2^{-p_z}}{2} \\
& \Leftrightarrow
  2^{-p_\op} \le \frac{2^{-p_z}}{2|\hat{z}|} \\
& \Leftrightarrow
  2^{-p_\op}\le 2^{-(p_z+2+\ld|\hat{z}|)} \\
& \Leftrightarrow
  2^{-p_\op}\le 2^{-\left(p_z+2+\frac{\ld|\hat{x}|}{2}\right)}
\end{split} 
\end{equation*}

\begin{equation*}
\begin{split}
  \frac{|\hat{x}-x|}{\sqrt{\hat{x}}} & \le \frac{2^{-p_z}}{2} \\
& \Leftrightarrow
  |\hat{x}-x| \le 2^{-\left(p_z+2+\ld\sqrt{\hat{x}}\right)}
               =  2^{-\left(p_z+2+\frac{\ld\hat{x}}{2}\right)} \\
& \Leftrightarrow
   2^{-p_x}  \le  2^{-\left(p_z+2+\frac{\ld\hat{x}}{2}\right)}
\end{split} 
\end{equation*}

\begin{algorithm}
\phantom{}
\begin{center}
\renewcommand*{\arraystretch}{1.3}
\begin{tabular}{lcl}
  \hline
  \multicolumn{3}{c}{Quadratwurzel} \\
  \hline
   $p_x$   & $\defeq$ & $p_z+2+\frac{\ld\hat{x}+1}{2}$ \\
   $p_\op$ & $\defeq$ & $p_z+2+\frac{\ld\hat{x}+1}{2}$ \\
  \hline
\end{tabular}
\end{center}
\end{algorithm}







\pagebreak
\section{$n$-te Wurzel}

\renewcommand{\op}{{\sqrt[n]{\phantom{\cdot}}}}

Berechnung einer Näherung $\hat{z}$ von $z=\sqrt[n]{x}$ mit $p_z$ exakten Stellen.

\begin{equation*}
\begin{split}
  |\hat{z}-z| & =  |\hat{z}-\sqrt[n]{x}| \\
              & =  |\hat{z}-\sqrt[n]{\hat{x}}+\sqrt[n]{\hat{x}}-\sqrt[n]{x}| \\
              &\le |\hat{z}-\sqrt[n]{\hat{x}}|+|\sqrt[n]{\hat{x}}-\sqrt[n]{x}| \\
              &\le |\hat{z}-\sqrt[n]{\hat{x}}|+|\hat{x}-x| \\
              &\le \frac{2^{-p_z}}{2} + \frac{2^{-p_z}}{2} \\
              & =  2^{-p_z}
\end{split}
\end{equation*}

\begin{equation*}
\begin{split}
  |\hat{z}-\sqrt[n]{\hat{x}}| \le \frac{2^{-p_z}}{2}
  & \Leftrightarrow
  \hat{z}\cdot 2^{-p_\op} \le 2^{-(p_z+2)}  \\
  & \Leftrightarrow
  2^{-p_\op}\le 2^{-(p_z+2+\ld\hat{z})} 
  = 2^{-\left(p_z+2+\frac{\ld\hat{x}}{n}\right)}
\end{split}
\end{equation*}

\begin{equation*}
  |\hat{x}-x| \le \frac{2^{-p_z}}{2} 
  \quad\Leftrightarrow\quad
  \hat{x}\cdot 2^{-p_x} \le 2^{-(p_z+2)}
  \quad\Leftrightarrow\quad
  2^{-p_x} \le 2^{-(p_z+2+\ld\hat{x})}
\end{equation*}

\begin{algorithm}
\phantom{}
\begin{center}
\renewcommand*{\arraystretch}{1.3}
\begin{tabular}{lcl}
  \hline
  \multicolumn{3}{c}{$n$-te Wurzel} \\
  \hline
   $p_x$   & $\defeq$ & $p_z+2+\ld\hat{x}$ \\
   $p_\op$ & $\defeq$ & $p_z+2+\frac{\ld\hat{x}+n-1}{n}$ \\
  \hline
\end{tabular}
\end{center}
\end{algorithm}

%
% EOF
%

\vfill\appendix
%
% LaTeX
%
\begin{thebibliography}{----}


\bibitem[1]{BFMSS}
Burnikel, C., Mehlhorn, K., et. Schirra, S.(1996):
\textit{“The LEDA class real number”}
\textit{(MPI-I-1996-1-001)},


\end{thebibliography}
%
% EOF
%


\end{document}
%
% EOF
%