%
% LaTeX
%
%\author{\Author}
\author{\Author\footnote{Dipl.-Ing. der Elektrotechnik, RWTH Aachen.}}
\date{\Date\footnote{\Rev}}
\maketitle
\vfill

\begin{center}
\bf Zusammenfassung
\end{center}

\noindent Der exakte Vergleich algebraischer Zahlen kann verhindern,
daß geometrische Algorithmen
aufgrund von Rundungsfehlern Fehlentscheidungen 
mit chaotischen Folgen treffen.
Um das Vorzeichen $\sgn\xi_E$
des exakten Wertes $\xi_E\in\A$
eines algebraischen Ausdrucks $E$
mit Operatoren $\in\{+, -, \times, \div, \sqrt[n]{\phantom{x}}\}$
und Operanden $\in\Q$ zu bestimmen,
berechnen einschlägige Algorithmen
eine Nulltrennungsschranke $\sep E$, die es erlaubt,
das Vorzeichen an einer hinreichend genauen
Näherung $\hat{\xi}_E$ abzulesen.
Wir schlagen vor, den Vorzeichentest für $E$ 
anhand eines Testausdrucks 
$E'$ mit $\sgn E=\sgn E'$ durchzuführen,
der zu einer speziellen Klasse wohlgeratener Ausdrücke gehört,
für deren Tests nicht unfaßbar viele Nachkommastellen
berechnet werden müssen,
so daß auch die Verarbeitung von Ausdrücken
mit vielen Wurzeloperationen praktikabel wird.
Eine Implementation in Java findet sich auf
github\footnote{\href{https://github.com/bestrauss/yaxana}{github.com/bestrauss/yaxana}}.


\begin{center}
\bf Abstract
\end{center}

\noindent The exact comparison of algebraic numbers can prevent
geometric algorithms from taking wrong decisions
due to rounding errors leading to chaotic program flow.
To compute the sign of the exact value $\xi_E\in\A$
of an algebraic expression $E$
with operators $\in\{+, -, \times, \div, \sqrt[n]{\phantom{x}}\}$
and operands $\in\Q$
pertinent algorithms calculate a root separation bound $\sep E$
which permits to read the sign 
from a sufficiently precise approximation $\hat{\xi}_E$.
We propose to perform the sign test of $E$
using a test expression $E'$ with $\sgn\xi_E=\sgn\xi_{E'}$
which belongs to a special class of well worked expressions
whose tests do not require the computation of inconceivably 
many $k$-ary places.
Thus, the processing of expressions with many root operations
becomes feasible.
An implementation in Java is available on github\footnotemark[1].

\vfill

%
% EOF
%
