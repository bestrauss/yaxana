%
% LaTeX
%
\pagebreak
\section{Definitionen}

\begin{definition}[Abstrakter Syntaxbaum, Terminal]
Ein Ausdruck $E$ ist ein Baum 
im Sinne der Graphentheorie.
Die Blätter des Baums sind Operanden, 
die restlichen Knoten unäre oder binäre Operatoren.
Im Sinne der Automatentheorie und der Linguistik
handelt es sich um einen abstrakten Syntaxbaum,
dessen Blätter terminale Symbole bzw. Terminale 
genannt werden.

\begin{note}
Ein Ausdruck kann auch als ein 
(gegenüber dem Baum allgemeinerer)
gerichteter, azyklischer Graph dargestellt werden,
indem identische Unterausdrücke im Baum 
durch nur ein Exemplar ersetzt werden,
auf das dann mehr als eine der gerichteten Kanten zeigt.
Dies wird in Implementationen genutzt,
ist aber im Rahmen dieser Ausführungen nicht weiter von Bedeutung.
\end{note}
\end{definition}


\pagebreak
\begin{definition}[Ausdruck]\label{de:Ausdruck}
Ein Ausdruck $E$ mit dem Wert $\xi_E$ ist:
\begin{itemize}
\item ein Terminal mit dem Wert $c\in\Q, c>0$,
\item die Summe $E_0+E_1$ zweier Ausdrücke,
\item das Produkt $E_0\times E_1$ zweier Ausdrücke,
\item die Negation $-E$ eines Ausdrucks,
\item der Kehrwert $1\div E$ eines Ausdrucks,
\item die $n$-te Wurzel $\sqrt[n]{E}$ eines Ausdrucks,
\item der Absolutwert $|E|$ eines Ausdrucks,
\end{itemize}
sofern
\begin{itemize}
\item beim Kehrwert $\xi_E\ne 0$,
\item bei der $n$-ten Wurzel $n\in\N\backslash\{1\}$ 
	        und, bei geradem $n$, $\xi_E\ge 0$.
\end{itemize}
Unterausdrücke mit verschwindenden Werten
werden detektiert und wegoptimiert.
Alle Radikanden werden positiv gemacht,
$\sqrt[n]{-E}=-\sqrt[n]{E}$, falls $\xi_E$ ungerade.
\begin{note}
Die Auswahl der Operatoren dient der Übersichtlichkeit
der folgenden Darstellungen und der Beweise.
Subtraktion, Division 
und Potenz mit beliebigen rationalen Exponenten können
auf diese Auswahl zurückgeführt werden.
Eine andere gebräuchliche Auswahl an Operatoren
ist $\in\{+, -, \times, \div, \sqrt[n]{\phantom{x}}\}$.
\end{note}
\end{definition}


\begin{definition}[Wert eines Ausdrucks]
Der Zahlenwert oder auch einfach 
Wert $\xi_E=\val{E}\in\A$ eines Ausdrucks $E$ ist
\begin{itemize}
\item $\xi_{c            } = c$
\item $\xi_{E_0+      E_1} = \xi_{E_0} + \xi_{E_1}$
\item $\xi_{E_0\times E_1} = \xi_{E_0} \times \xi_{E_1}$
\item $\xi_{1  \div   E  } = 1 \div \xi_E$
\item $\xi_{-E           } = -\xi_{E}$
\item $\xi_{\sqrt[n]{E}  } = \sqrt[n]{\xi_E}$
\item $\xi_{|E|          } = |\xi_E|$
\end{itemize}
\end{definition}


\pagebreak
\begin{definition}[Strukturpolynom]\label{de:Strukturpolynom}
Das Strukturpolynom $p_E(z)$ eines Ausdrucks $E$
mit dem Wert $\xi_E$
ist ein Polynom mit rationalen Koeffzienten,
das eine Nullstelle hat, deren Betrag $|\xi_E|$ ist
und das anhand des abstrakten Syntaxbaums von $E$
nach folgenden Bildungsregeln konstruiert wird:
\begin{equation*}
\begin{split}
p_{c            }(z) & = z-c \\
p_{E_0+      E_1}(z) & = z^{-K}\prod_{k_0=0}^{N_0}\prod_{k_1=0}^{N_1} (z-(z_{k_0}+z_{k_1})) \\
p_{E_0\times E_1}(z) & = \prod_{k_0=0}^{N_0}\prod_{k_1=0}^{N_1} (z-(z_{k_0}\times z_{k_1}))\\
p_{1  \div   E  }(z) & = \prod_{k=1}^N (z-(1\div z_k)) \quad,\quad
p_{\sqrt[n]{E}  }(z) & = \prod_{k=1}^N (z^n-z_k) \\
p_{-E}(z) & = p_{|E|}(z)  = p_E(z) = \prod_{k=1}^N (z-z_k)\\
\end{split}
\end{equation*}
Dabei
\begin{itemize}
\item sind $c, z_k, z_{k_0}$ bzw. $z_{k_1}$ auf der rechten Seite 
      jeweils die Nullstellen 
      von $p_c(z), p_E(z), p_{E_0}(z)$ bzw. $p_{E_1}(z)$,
      die nicht verschwinden, denn die Bildungsregeln 
	  gewährleisten dies.
\item haben die gebildeten Polynome rationale Koeffizienten,
      was ebenfalls durch die Bildungsregeln gewährleistet ist.
      Dies wird abgesehen von den trivialen Fällen
	  in \refappendix{se:Anhang} gezeigt.
\item könnte bei der Addition zweier Nullstellen $z_{k_0}+z_{k_1}$
      eine Nullstelle bei $z=0$ entstehen,
	  was aber durch den Term $z^{-K}$ vereitelt wird,
	  der bis zu $K$ Nullstellen bei $z=0$ aufhebt.
	  $K$ wird als Anzahl der Nullstellen gewählt,
      die das Polynom ohne den Term $z^{-K}$ hätte.
\item verändert der Term $z^{-K}$ die Koeffizienten nicht,
      sondern tauscht sie nur aus (verschiebt sie), 
	  weshalb sie gleich und damit weiterhin rational bleiben.
\item liegen vor Einsetzen des Terms $z^{-K}$ 
      niemals sämtliche Nullstellen bei $z=0$,
      da per \refdefinition{de:Ausdruck} 
      die Werte von Unterausdrücken nicht verschwinden,
      außer möglicherweise an der Baumwurzel 
	  des abstrakten Syntaxbaums,
      wo ein solches Ereignis dann $\xi_E=0$ anzeigt.
\end{itemize}
\begin{note}
Es heißt oben, das (monströse) Strukturpolynom werde konstruiert.
Tatsächlich dient das Strukturpolynom nur theoretischen Betrachtungen
und kommt in der Praxis, im Algorithmus, nicht vor.
\end{note}
\end{definition}


\begin{remark}[Struktur- vs. Minimalpolynom]
Im Gegensatz zum Minimalpolynom einer algebraischen Zahl
kodiert das Strukturpolynom eines Ausdrucks
wesentliche Aspekte der Struktur des Ausdrucks
inklusive der Werte seiner Terminale
und der Wurzelexponenten.
Bildet man aus zwei Ausdrücken mit Werten,
deren Minimalpolynome hochgradig sind,
einen neuen Ausdruck, 
dessen Wert rational ist,
dann findet ein Kollaps statt und
das Minimalpolynom des Wertes des neuen Ausdrucks
hat nur noch den Grad $1$.
Das ist bei Strukturpolynomen nicht der Fall.
Unabhängig vom Wert des Ausdrucks reflektiert
der Grad des Strukturpolynoms 
die Komplexität des Ausdrucks.
Aus diesem Grund sind Minimalpolynome hier zuwider,
von Interesse ist nicht der Begriff
,,konjugierte algebraische Zahl``
sondern der Begriff ,,konjugierter Ausdruck``.
\end{remark}


\begin{definition}[Konjugierter Ausdruck]\label{de:Konjugierter Ausdruck}
Zwei Ausdrücke $E_0$ und $E_1$ heißen {\em konjugiert},
wenn sie ein und dasselbe Strukturpolynom 
$p_{E_0}(z)=p_{E_1}(z)$ haben.
\begin{note}
Zwei Ausdrücke $E_0$ und $E_1$,
die \underline{nicht} konjugiert sind, 
können denselben Wert $\xi_{E_0}=\xi_{E_1}$
mit demselben Minimalpolynom haben.
\end{note}
\end{definition}
