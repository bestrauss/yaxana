%
% LaTeX
%
\pagebreak
\section{The precision of an expression}

Let $E$ be an expression consisting of zero or more operators
$\in\{+, -, \times, \div, \sqrt[n]{\phantom{x}}\}$ and one or 
more operands $\in\Q$. Then any expression $E$ can be constructed 
in the following way. Starting with $E_0=c, c\in\Z$ 
we build $E_1$ from $E_0$, or generally $E_n$ from $E_{n-1}$ by
exchanging one suitably chosen operand $c$ by a new instance 
of one of the subexpressions
$c+d$, $c-d$, $c\times d$, $c\div d$, or $\sqrt c$. 
Using a suitably chosen series of operands $c$ to be replaced,
suitably chosen operators and new operands $d$, 
any expression $E$ can be constructed in this way.

Let $f$ be a function $f(E)$ that, given an expression $E$,
returns a precision of an approximation $\hat\xi$
of the exact value $\xi=\val E$, 
that is sufficient to ensure $\sgn\hat\xi=\sgn\xi$.
If $f(E)$ works for $E=E_0$, 
and given that $f(E_n)$ works
we can imply that $f(E_{n+1})$ works,
then by induction $f(E)$ works for any expression $E$.

We define the {\em point-length} of a constant $c\in\Q$ 
looking at the binary fixed point representation 
of the value.
We look for the smallest coherent block of bits 
that includes all non-vanishing bits as well as the
binary point. The {\em point-length} then is the
number of bits contained in the block (the binary point
not counting as a bit).

Proof of the above mentioned implications.
\begin{enumerate}
\item Initial step $E_0=c$.
      $f(E)$ returns the total length of $c$ as precision
	  which suffices to ensure $\sgn\hat\xi=\sgn\xi$.
\item Replacing $c$ by $c\rightarrow d$.
      If $f(E)$ works for arbitrary values $c$
	  then $f(E)$ works for arbitrary values $d$, too.
\item Replacing $c$ by $c\rightarrow c\pm d$.
      We divide this step into two.
	  First we replace $c\rightarrow e$ choosing $e$ such that $e = c\pm d$.
	  Then we replace $e$ by $c\pm d$.
	  The first step has been proven correct above in 2.).
	  The second steps leaves $\val E_n$ unchanged $\val E_{n+1}=\val E_n$,
	  while $f(E)$ returns the same precision for $e$ and for $c\pm d$.
	  This is, because $f(e) = f(c\pm d) = max(f(c), f(d))$.
\item Replacing $c$ by $c\rightarrow c\times d$.
      We divide this step into two.
	  First we replace $c\rightarrow e$ choosing $e$ such that $e = c \times d$.
	  Then we replace $e$ by $c\times d$.
	  The first step has been proven correct above in 2.).
	  The second steps leaves $\val E_n$ unchanged $\val E_{n+1}=\val E_n$,
	  while $f(E)$ returns the same precision for $e$ and for $c\times d$.
	  This is, because $f(e) = f(c\times d) = f(c) + f(d)$.
\item Replacing $c$ by $c\rightarrow c\div d$.
      {\bf TODO} $\ldots$
\item Replacing $c$ by $c\rightarrow \sqrt c$.
      {\bf TODO} $\ldots$
\end{enumerate}


What precision does an approximation $\hat\xi$ of $\xi=\val E$ need to have to ensure that we can read from $\hat\xi$ whether $\xi\in\Q$?

We don't admit any rational number. We allow {\em binary rationals} only,
that is, quotients of integer nominators and power of two denominators.
We represent them as $m\times 2^k, m, k\in\Z$.

If don't admit any rational root, but only square-roots, or $2^k$-roots,
then we can easily remove all non-real polynomial roots from the definition
of the structural polynomial.
Then $E=\sqrt[256]{2^{256}+1}-2$ 
has 2 roots only whose product is $-\xi(4+\xi)\in\Q$
and whose sum is $\in\Z$.

Das Produkt der konjugierten Paare ist immer reell.