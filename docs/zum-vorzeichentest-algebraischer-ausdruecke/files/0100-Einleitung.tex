%
% LaTeX
%
\pagebreak
\section{Einleitung}\label{se:Einleitung}


\begin{flushright}
\begin{minipage}{6.85cm}
{\em ,,Omnia in mensura, et numero et pondere disposuisti.``} \hfill
\scriptsize{Vulgata, Liber Sapientiæ 11:21}
\end{minipage}
\end{flushright}


\noindent
Gegeben sei ein Ausdruck $E$,
dessen Wert eine reelle algebraische Zahl $\xi_E=\val_E\in\A$ ist.
$E$ verknüpft positive rationale Zahlen
mittels der Operatoren $\in\{+, -, \times, \div, \sqrt[n]{\phantom{x}}\}$
Beispiel:
\begin{equation}\label{eq:ausdruck}
  \xi_E = \sqrt{2} + \sqrt{3} - \sqrt{5 + 2\times\sqrt{6}}.
\end{equation}
Eine exakte Darstellung des Zahlenwerts $\xi_E = \textrm{val}(E)$ von $E$ 
im Binär- oder Dezimalsystem oder dergleichen ist 
aufgrund endloser Ziffernfolgen i.A. nicht möglich.

Andererseits ist es möglich, fehlerfreie exakte Werte 
für das Vorzeichen $\sgn{\xi_E}$ von $\xi_E$ zu bestimmen. 
Der exakte Vergleich algebraischer Zahlen kann sicherstellen,
daß etwa geometrische Algorithmen
frei von Rundungsfehlern exakte Entscheidungen treffen
und der Programmfluß nicht vom Chaos beherrscht wird.

Mittels der
in \cite{Mignotte, BFMS, BFMSS, PIYAP} beschriebenen Verfahren
läßt sich eine Nulltrennungsschranke $\sep E>0$ berechnen, 
so daß entweder $\xi_E=0$ oder $|\xi_E|\ge\sep E$ gilt.  %\veebar
Die Nulltrennungsschranke $\sep E$ garantiert dann, 
daß der Wert von $\sgn{\xi_E}$
an einer entsprechend genauen Näherung $\hat{\xi}_E$ für $\xi_E$
abgelesen werden kann.
Die Verfahren sind robust, 
berechnen aber im Fall $\xi_E=0$ 
bei Ausdrücken mit mehr als ein paar Wurzeloperationen
unfaßbar viele Nachkommastellen,
deren Anzahl mit dem Produkt der Wurzelexponenten
der Wurzeloperationen in $E$ wächst.

Zwecks Verbesserung der Situation wird vorgeschlagen,
den Vorzeichentest für $E$ 
anhand eines mit $E$ verwandten Testausdrucks $E'$ 
mit $\sgn\xi_E=\sgn\xi_{E'}$ durchzuführen,
so daß gewisse Eigenschaften der Nullstellenverteilung 
des Strukturpolynoms von $E'$ die Verwendung
vergleichsweise traumhafter Nulltrennungsschranken erlaubt.
Der Rechenaufwand im kritischen Fall $\xi_E=0$
wird für Ausdrücke mit vielen unterschiedlichen Wurzeloperationen 
drastisch reduziert bzw. der Test solcher Ausdrücke
überhaupt erst praktikabel gemacht.

Es folgen Abschnitte mit Definitionen
und mit einer Vorzeichentesttheorie.
Danach Fazit 
sowie im Anhang Beweise zum Strukturpolynom
und Literaturverzeichnis.

%
% EOF
%
